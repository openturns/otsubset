% 
% Permission is granted to copy, distribute and/or modify this document
% under the terms of the GNU Free Documentation License, Version 1.2
% or any later version published by the Free Software Foundation;
% with no Invariant Sections, no Front-Cover Texts, and no Back-Cover
% Texts.  A copy of the license is included in the section entitled "GNU
% Free Documentation License".

%%%%%%%%%%%%%%%%%%%%%%%%%%%%%%%%%%%%%%%%%%%%%%%%%%%%%%%%%%%%%%%%%%%%%%%%%%%%%%%%%%%%%%%%%% 
\section{User Manual}

This section gives an exhaustive presentation of the objects and functions provided by the $subset$ module, in the alphabetic order.

\subsection{SubsetSampling}

This class inherits from the Simulation class, please refer to the Simulation class documentation for common methods.\\
Note that this algorithm cannot be controlled by means of a maximum coefficient of variation.\\

\begin{description}

  \item[Usage :] \rule{0pt}{1em}
    \begin{description}
    \item $SubsetSampling(event)$
    \item $SubsetSampling(event, proposalRange)$
    \item $SubsetSampling(event, proposalRange, targetProbability)$
    \end{description}

  \item[Arguments :]  \rule{0pt}{1em}
    \begin{description}
    \item $event$ : an Event, the event we want to compute the probability of
    \item $proposalRange$ : a NumericalScalar, the range of the uniform proposal as distribution of the random walk (default$=2.0$)
    \item $targetProbability$ : a NumericalScalar, the probability estimate of each conditional domain $P(F_i|F_{i-1})$ (default$=0.1$)
    \end{description}

  \item[Value :] a SubsetSampling

  \item[Details :]  \rule{0pt}{1em}
    \begin{description}
    \item Like all the simulation algorithms, the number of samples per step is determined by $maximumOuterSampling*blockSize$ (default $1000 \times 1$)
    \item Like usual, use the $getResult$ method to access the computed results.
    \end{description}

  \item $getProposalRange$
    \begin{description}
    \item[Usage :] $getProposalRange()$
    \item[Arguments :] no argument
    \item[Value :] a NumericalScalar, the range of the uniform proposal as distribution of the random walk
    \end{description}
    \bigskip

  \item $getTargetProbability$
    \begin{description}
    \item[Usage :] $getTargetProbability()$
    \item[Arguments :] no argument
    \item[Value :] a NumericalScalar, the probability estimate of each conditional domain $P(F_i|F_{i-1})$
    \end{description}
    \bigskip

  \item $getNumberOfSteps$
    \begin{description}
    \item[Usage :] $getNumberOfSteps()$
    \item[Arguments :] no argument
    \item[Value :] an UnsignedLong, the number of subset steps
    \end{description}
    \bigskip

  \item $getThresholdPerStep$
    \begin{description}
    \item[Usage :] $getThresholdPerStep()$
    \item[Arguments :] no argument
    \item[Value :] a NumericalPoint, the intermediate threshold values
    \end{description}
    \bigskip

  \item $getGammaPerStep$
    \begin{description}
    \item[Usage :] $getGammaPerStep()$
    \item[Arguments :] no argument
    \item[Value :] a NumericalPoint, the intermediate autocorrelation values
    \end{description}
    \bigskip  
    
   \item $getCoefficientOfVariationPerStep$
    \begin{description}
    \item[Usage :] $getCoefficientOfVariationPerStep()$
    \item[Arguments :] no argument
    \item[Value :] a NumericalPoint, the intermediate c.o.v. values
    \end{description}
    \bigskip  
      
  \item $getProbabilityEstimatePerStep$
    \begin{description}
    \item[Usage :] $getProbabilityEstimatePerStep()$
    \item[Arguments :] no argument
    \item[Value :] a NumericalPoint, the intermediate pf values
    \end{description}
    \bigskip 
    
  \item $setKeepEventSample$
    \begin{description}
    \item[Usage :] $setKeepEventSample(keepEventSample)$
    \item[Arguments :] a boolean keepEventSample, deciding whether we keep the event samples
    \item[Value :] none
    \end{description}
    \bigskip   
    
  \item $getEventInputSample$
    \begin{description}
    \item[Usage :] $getEventInputSample()$
    \item[Arguments :] no argument
    \item[Value :] a NumericalSample, the input sample values. See $setKeepEventSample$.
    \end{description}
    \bigskip 
    
  \item $getEventOutputSample$
    \begin{description}
    \item[Usage :] $getEventOutputSample()$
    \item[Arguments :] no argument
    \item[Value :] a NumericalSample, the output sample values. See $setKeepEventSample$.
    \end{description}
    \bigskip 
    
The methods $getProposalRange$, $getTargetProbability$ have their corresponding $setMethod$.

\item[Links] \rule{0pt}{1em}
\end{description}


\newpage \subsection{SubsetSamplingResult}

This class inherits from the SimulationResultImplementation class, please refer to the SimulationResult class documentation for common methods, \dots \\
Note that the $outersampling$ result can exceed the algorithms $maximumOuterSampling$, as the total number of samples is determined by the number of steps, so that 
$outersampling = maximumOuterSampling \times numberOfSteps$.\\

