% 
% Copyright 2009-2012 Phimeca

%%%%%%%%%%%%%%%%%%%%%%%%%%%%%%%%%%%%%%%%%%%%%%%%%%%%%%%%%%%%%%%%%%%%%%%%%%%%%%%%%%%%%%%%%% 



\section{Reference Guide}

\subsection{Subset Simulation}

\MathematicalDescription{
\underline{\textbf{Goal}}\\
Let us note $\mathcal D_f = \{\ux \in \mathbb R^{n_X} | g(\ux,\underline{d}) \leq 0\}$.
The goal is to estimate the following probability:
\begin{eqnarray*}
        P_f &=& \int_{\mathcal D_f} f_{\uX}(\ux)d\ux\\
            &=& \int_{\mathbb R^{n_X}} \mathbf{1}_{\{g(\ux,\underline{d}) \:\leq 0\: \}}f_{\uX}(\ux)d\ux\\
            &=& \Prob {\{g(\uX,\underline{d}) \leq 0\}}
\end{eqnarray*}

\underline{\textbf{Principles}}\\
 The idea of the Subset Sampling method is to replace simulating a rare failure event in the original probability space by a sequence of simulations of more frequent conditional events $F_i$ 
\begin{eqnarray*}
  F_1 \supset F_2 \supset \dots \supset F_m = F
\end{eqnarray*}
The original probability estimate rewrites
\begin{eqnarray*}
  P_f = P(F_m) = P(\bigcap \limits_{i=1}^m F_i) = P(F_1) \prod_{i=2}^m P(F_i|F_{i-1})
\end{eqnarray*}
And each conditional subset failure region is chosen by setting the threshold $g_i$ so that $P(F_i|F_{i-1})$ leads to a conditional failure probability of order $0.1$\\
\begin{eqnarray*}
  F_i =\Prob {\{g(\uX,\underline{d}) \leq g_i\}}
\end{eqnarray*}
The conditional samples are generated by the means of Markov Chains, using the Metropolis Hastings algorithm. 

\underline{\textbf{Coefficient of variation}}\\
$N$ being the number of simulations per subset, and $p_{0i}$ the conditional probability of each subset event, and $\gamma_i$ the autocorrelation between Markov chain samples.
\begin{eqnarray*}
  \delta^2 = \sum_{i=1}^m \delta^2_i = \sum_{i=1}^m (1+\gamma_i) \frac{1-p_{0i}}{p_{0i}N}
\end{eqnarray*}
The first event $F_1$ not being conditional, $\delta^2_1$ expresses as the classic Monte Carlo c.o.v.
}
{
% Autres notations et appellations
}

\Methodology{
This method is part of the step C of the global methodology. It requires the specification of the joined probability density function of the input variables and the value of the threshold and the comparison operator.\\
}
{
$p0$ being the same conditional probability for each level, the number of steps is given by 
\begin{eqnarray*}
   m = \frac{\log P_f}{\log p0}
\end{eqnarray*}
The total number of samples required to reach a c.o.v. of $\delta$ in the global probability estimate can be approximated by
\begin{eqnarray*}
  N_T \approx m N = |\log P_f|^r \frac{(1+\gamma)(1-p_0)}{p_0|\log p_0|^r \delta^2}
\end{eqnarray*}

So the subset simulation shows a substantial improvement ($N_T \sim \log P_f$) compared to crude Monte Carlo ($N_T \sim \frac{1}{P_f}$) sampling when estimating rare events.

\vspace{10mm}
The following references are a first introduction to the subset simulation method:\\

\label{ab01} S.K. Au and J. L. Beck \textit{Estimation of small failure probabilities in high dimensions by subset simulation}. In Probabilistic Engineering Mechanics 16, 2001

}
